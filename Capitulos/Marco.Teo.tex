\chapter{MARCO TEÓRICO}
La política monetaria es un instrumento de la macroeconomía
\section{Referencias}
La famosa monografía de  \cite{friedman2018theory}  constituye su aportación a la teoría del consumo. 

Desde el punto de vista del viejo keynesianismo, la reformulación que hace \cite{friedman2018theory} de la teoría de la cantidad de dinero es plenamente aceptable, sólo se vuelve problemática cuando posteriormente se la coloca en el contexto del monetarismo. De hecho, la formulación que hace Friedman de la demanda de dinero como parte de un programa general de maximización de rendimientos corrige una especificación de flujo muy importante en el modelo keynesiano is-lm \cite{hicks1937mr}. keynesiana de tasas de interés basada en preferencia de liquidez que quedó incorporada en la formulación que hace \cite{tobin1982money} del modelo is-lm keynesiano de múltiples activos

\section{Marcos}

\begin{tcolorbox}
La economía es una ciencia social que estudia la forma de administrar los recursos disponibles para satisfacer las necesidades humanas. Además, también estudia el comportamiento y las acciones de los seres humanos.
\end{tcolorbox}



\begin{tcolorbox}[title=\textbf{Ejemplos},
colback=blue!5!white,colframe=blue!75!white]
La economia es una ciencia que estudia :
\tcblower
La forma de administrar los recursos disponibles para satisfacer las necesidades humanas. Además, también estudia el comportamiento y las acciones de los seres humanos.
\end{tcolorbox}




\begin{table}
\centering
\caption{Diferencias... entre \LaTeX\ clases}
\renewcommand{\arraystretch}{1.6}
\begin{tabular}{lccccc}
\toprule
Tesis & Intr.&Marco Teórico&Marco Legal &
Marco Demostra. & Conclusión \\
\cmidrule(r){1-1}\cmidrule(lr){2-2}\cmidrule(lr){3-3}
\cmidrule(lr){4-4}\cmidrule(lr){5-5}\cmidrule(l){6-6}
Arti. & & & \Checkmark & \\
Libro& \Checkmark & \Checkmark & &
\Checkmark & \Checkmark \\
Reporte & \Checkmark & \Checkmark & \Checkmark &
& \Checkmark \\
\bottomrule
\end{tabular}

\label{comparison}
\end{table}









\section{Matemática}


\[
\int_a^b f(x)\,\mathrm{d}x \approx (b-a)
\sum_{i=0}^n w_i f(x_i)
\]


\begin{equation*}
\begin{split}
(a+b+c+d+e+f)^2 & = a^2+b^2+c^2+d^2+e^2+f^2\\
&\quad +2ab+2ac+2ad+2ae+2af\\
&\quad +2bc+2bd+2be+2bf\\
&\quad +2cd+2ce+2cf\\
&\quad +2de+2df\\
&\quad +2ef
\end{split}
\end{equation*}


%\begin{multline}
%5x_1 + 2x_2 + 3x_3 -\\
%x_4 - 4x_5 + 5x_6 +\\
%7x_7 + 3x_8 - 6x_9 -\\
%2x_{10} - 5x_{11} = 7634
%\end{multline}
%
%
%\begin{multline}
%5x_1 + 2x_2 + 3x_3 -\\
%x_4 - 4x_5 + 5x_6 +\\
%7x_7 + 3x_8 - 6x_9 -\\
%2x_{10} - 5x_{11} = 7634
%\end{multline}



