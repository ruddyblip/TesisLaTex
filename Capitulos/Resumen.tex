\begin{Resumen}
La presente tesis tiene como objetivo general investigar el efecto que genera la Bolivianización, el Índice del tipo de cambio real, la Remesas y las Exportaciones 


%(excluyendo los ingresos por exportación del gas natural) sobre la Fluctuación de las Reservas Internacionales Netas del Estado Plurinacional de Bolivia en el periodo 2006 – 2021, para lo cual se elaboró la siguiente hipótesis de trabajo “El comportamiento fluctuante de las Reservas Internacionales del Estado Plurinacional de Bolivia es resultado de factores internos ligados a la política monetaria y cambiaria implementada a partir del año 2006 (Bolivianización e índice del tipo de cambio real), y a factores del contexto internacional (Remesas y Exportaciones).
\end{Resumen}