\chapter{MARCO REFERENCIAL METODOLÓGICO}
\section{Delimitación General}
Las dos categorías económicas que enmarcarán los límites de la investigación son las Inversiones Financieras, representado por los volúmenes negociados en bolsa, su tipificación y las tasas de interés que ofrecen; y el Mercado de Valores, representado por ofertantes y demandantes dentro de este mercado. Ambas categorías económicas estarán enmarcadas dentro de la microeconomía, sin desconocer sus alcances a nivel macroeconómico.

\subsection{Espacial}
hola como estas

Las entidades del sector financiero se encuentran particularmente expuestas al riesgo de liquidez, dada la naturaleza de sus actividades, entre las que se incluye la captación de fondos. 

Captar recursos financieros de los agentes económicos excedentarios para luego prestarlos a los agentes económicos deficitarios; esta importante labor, facilita las transacciones y el flujo de dinero hacia sectores productivos de la economía.

\subsection{Sectorial}
Los sectores económicos involucrados son el sector financiero, especialmente el del mercado de capitales y el mercado monetario, el sector productivo y el estado.
\subsection{Institucional}

El Mercado de capitales, la bolsa, como intermediario directo actúa en relación con los siguientes actores: El Estado, que propone y diseña políticas a través del Ministerio de Economía y Finanzas Públicas y el Banco Central de Bolivia (que además funge como inversionista y emisor por su carácter autárquico).
 Las instituciones de regulación, que son la Autoridad de Supervisión y control del Sistema Financiero (ASFI) y la Autoridad de Fiscalización y Control de Pensiones y Seguros (APS).
 

\subsection{Delimitacion en la Mención}
La Economía Financiera, por definición, abarca todo el espectro del análisis del Sistema Financiero, sus instituciones y las familias, y de cómo éstos asignan sus recursos en un entorno incierto. La bolsa es una institución que representa por excelencia esta definición, fungiendo como un canal sumamente importante en las decisiones de inversión de los agentes económicos y supliendo las necesidades de recursos de las empresas.
\section{Restricción de Variables Económicas}
\subsection{Categorías Económicas}
A continuación mostraré más ejemplos  
\begin{description}
    \item[C.E.1] Inversión Financiera.
    \item[C.E.2] Mercado Financiera. 
\end{description}
\subsection{Variables Económicas}
A continuación mostraré más ejemplos
\begin{description}
    \item[V.E.1.1.] Inversión Financiera.
    \item[V.E.1.2.] Mercado Financiera.
\end{description}
Existen métodos para manipular las etiquetas, pero para describirlos necesitamos conocimientos relativamente avanzados, por lo que los abordaremos más adelante. 

\section{Planteamiento del Objeto de Investigación}
\section{Planteamiento del Problema}
Para la formulación del problema, debemos ir de lo general a lo particular, pues se parte de una interrogante que engloba un problema que luego irá siendo abordado por partes.

\subsection{Objetivo Especifico}
\begin{description}
    \item[O.E. 1.1. Evaluar]  la evolución del Volumen Total de Inversiones y el Tipo de Operación.
    \item[O.E. 1.2. Comparar]  las inversiones por Tipo de Instrumento con el Total de Inversiones.
    \item[O.E. 1.3. Comparar]  las inversiones por Tipo de Instrumento con el Total de Inversiones.
\end{description}

\section{Fuente de Información }
Para la 
\begin{itemize}
  \item Instituto Nac.ional de Estadistica (INE)
  \item Fundación Jubileo.
\item Fundación Milenio.
\end{itemize}